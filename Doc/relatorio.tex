\documentclass[a4paper,12pt,fleqn]{article}
%\usepackage {psfig,epsfig} % para incluir figuras em PostScript
\usepackage{amsfonts,amsthm,amsopn,amssymb,latexsym,subcaption}
\usepackage{graphicx}
\usepackage[T1]{fontenc}
\usepackage[brazil]{babel}
\usepackage{geometry}
\usepackage[utf8]{inputenc}
\usepackage[intlimits]{amsmath}
\usepackage{multicol}

% formato de codigo
\usepackage{listings}
\lstset{
 	breaklines=true,
	basicstyle=\scriptsize,
	language=C++,
	numbers=left,
	numbersep=-10pt,
	stepnumber=1,
	tabsize=2,
	showstringspaces=false
}

%alguns macros
\newcommand{\R}{\ensuremath{\mathbb{R}}}
\newcommand{\Rn}{{\ensuremath{\mathbb{R}}}^{n}}
\newcommand{\Rm}{{\ensuremath{\mathbb{R}}}^{m}}
\newcommand{\Rmn}{{\ensuremath{\mathbb{R}}}^{{m}\times{n}}}
\newcommand{\contcaption}[1]{\vspace*{-0.6\baselineskip}\begin{center}#1\end{center}\vspace*{-0.6\baselineskip}}
%=======================================================================
% Dimensões da página
\usepackage{a4}                       % tamanho da página
\setlength{\textwidth}{16.0cm}        % largura do texto
\setlength{\textheight}{9.0in}        % tamanho do texto (sem head, etc)
\renewcommand{\baselinestretch}{1.15} % espaçamento entrelinhas
\addtolength{\topmargin}{-1cm}        % espaço entre o head e a margem
\setlength{\oddsidemargin}{-0.1cm}    % espaço entre o texto e a margem
       
% Ser indulgente no preenchimento das linhas
\sloppy
 

\begin{document}


\pagestyle {empty}

%\input logo

%\vspace*{-3cm}

%\begin{figure}[h]
%\leavevmode
%\begin{minipage}[t]{\textwidth}
%\includegraphics[1cm,1cm][3cm,3cm]{logo-ufrpe.bmp}
%\end{minipage}
%\end{figure}



\vspace*{-2cm}
{\bf
\begin{center}
{\large
\hspace*{0cm}Universidade de São Paulo} \\
\hspace*{0cm}Instituto de Matemática e Estatística \\
\hspace*{0cm}MAC 5742 - Computação Paralela e Distribuída  \\
\end{center}}
\vspace{4.0cm}
\noindent
\begin{center}
{\Large \bf Exercício Programa 1: OpenMP} \\[3cm]
{\Large Autores:}\\[6mm]
{\Large Diana Naranjo}\\[6mm]
{\Large Walter Perez}\\[6mm]
\end{center}




\vspace{8cm}
{\center São Paulo \\[3mm]
Abril 2015 \\}


\newpage


  
\pagestyle {empty}
\abstract{Nesse Exercício Programa (EP) o objetivo foi explorar a computação paralela com memoria compartilhada, para isso foi usado o padrão OpenMP.
A primeira parte do trabalho explora o cuidado que deve-se ter no momento de realizar o desenvolvimento de programas usando as diretivas do OpenMP. É muito simples cometer erros quando ainda se está pensando de maneira sequencial, ao assumir algum comportamento ou quando não se conhece bem o comportamento padrão das diretivas usadas.
A segunda parte do EP procura avaliar as melhoras (ou falta delas) no tempo de execução de um programa alvo, mult.c, que realiza a multiplicação de 2 matrizes. Para avaliar o desempenho da versão sequencial versus a paralela uma serie de experimentos foram realizados. Alguns deles involucraram a alteração do programa para criar distintas zonas paralelas e também a execução deles usando diferentes números de threads.}

\newpage 
 
\tableofcontents 
 
 
% Numeração em romanos para páginas iniciais (sumários, listas, etc) 
%\pagenumbering {roman} 
\pagestyle {plain} 
 
 
 
\setcounter{page}{0} \pagenumbering{arabic} 
 
 
 
 
 
 
  
\setlength{\parindent}{0in}  %espaço entre paragrafo e margem 
% Espaçamento entre parágrafos 
\parskip 5pt   
 \clearpage 
 
\section{Introdução} 
A computação paralela consiste em executar um conjunto de cálculos de maneira simultânea ~\cite{Gottlieb89}. O objetivo é diminuir o tempo de execução que exigem algumas aplicações, e.g. aquelas com fortes requerimentos de computo. Atualmente, existem computadores com múltiplos cores; além de isso também contam com tecnologias como hyper-threading, o que gera que o sistema operativo assuma que conta com uma maior quantidade de recursos de computo. Os recursos, por tanto, estão prontos para ser usados. Porém, para realmente obter uma melhoria no rendimento não é suficiente executar as aplicações em uma máquina multicore, é necessário que elas troquem seus algoritmos sequências por uma versão paralela. De outra maneira, não estaria-se fazendo uso do processamento paralelo e seus benefícios.  

O grande problema é que a troca de algoritmos sequências por paralelos não é um trabalho simples. Existem diversos problemas na geração de códigos paralelos de bom rendimento, alguns deles são ~\cite{Matloff14}: 
\begin{itemize} 
    \item Gargalos de comunicação. Debe ter-se em consideração a quantidade processadores que serão utilizados na execução e o overhead produto da comunicação entre estes processadores e a memoria~\cite{Gebali11}. Isto é importante pois pode acontecer que um programa em versão paralela tome mais tempo na execução do que na versão sequencial. O que se debe a que a memoria só pode ser acessada por um thread em qualquer momento e o custo de manter as caches de cada processador coerente também gera um overhead. 
    \item Balanceamento de carga. Ao fazer uma divisão de trabalho o ideal é que cada uma das partes tenha uma carga equivalente. Se uma delas tem mais trabalho do que as outras então no final o problema volta a ser sequencial pois a maioria terminam com suas cargas e umas poucas continuam com o resto do trabalho. 
    \item Problemas na construção do código. Os desenvolvedores estão acostumados a pensar em forma sequencial e por tanto é muito fácil cometer erros de concorrência. As operações simples, como por exemplo a assinação de valores a uma variável, causam resultados inesperados na versão paralela. Isto é devido as condições de corrida, i.e. quando o programa se desenvolve em um ordem diferente ao que o programador planejou. 
\end{itemize} 
 Para evitar os erros produzidos por o pouco cuidado do desenvolvedor e para maximizar o paralelismo dos algoritmos que são sequências existem conjuntos de diretivas que criam um código paralelo executável. OpenMP é um conjunto de diretivas para o compilar, rutinas de libraria e variáveis de entorno que podem ser usadas para a geração de paralelismo em códigos Fortran e C/C++ ~\cite{OpenMP13}. O objetivo é garantir o correto funcionamento dos programas e obter os benefícios do paralelismo. 
 
Neste trabalho exploramos os problemas persistentes no uso das diretivas do OpenMP. Tanto no desenvolvimento de código errôneo, como na geração de overhead pela geração de áreas paralelas ineficientes. Para isso desenvolvemos experimentos para obter o tempo promédio na execução de códigos gerados a partir de distintas áreas paralelas, em ambientes diferentes e com distintas quantidades de threads. 
 
\section{Exercício 1} 
\textbf{Escolher um código na internet que use as diretivas de compilação do OpenMP, esse código deve ser procurado nos respetivos tutoriais e manuais desse padrão de programação multiprocessamento. A ideia e encontrar erros nessas implementações fornecidas ou apresentadas nos tutoriais consultados. Apresente o código, aponte os problemas e descreva quais são as correções feitas para tirar o erro da aplicação.} 
 
\subsection{Código com erro} 
	O seguinte programa foi extraído do site do Rogerio ~\cite{Rogerio15} e tem como objetivo imprimir os valores de $A[ i ] = i * i$ no mesmo ordem da iteração, neste caso para os valores de $i$ a partir de 0 a 15.

	\begin{lstlisting} 
		#include <stdio.h>
		#include <omp.h>
		#define SIZE 16

		int main(){
			int A[ SIZE ] , i ;
			#pragma omp parallel for schedule( static , 2 ) num_threads( 4 ) ordered
			for( i = 0 ; i < SIZE ; i++){
				A[ i ] = i * i ;
				printf( "Th[ %d ]: %02d = %03d\n" , omp_get_thread_num() , i , A[ i ] ) ;
			}
			return 0 ;
		}
	\end{lstlisting}
	
	O resultado sería:
	\begin{multicols}{3}
		\begin{verbatim}
			Th[ 0 ]: 00 = 000
			Th[ 0 ]: 01 = 001
			Th[ 1 ]: 02 = 004
			Th[ 1 ]: 03 = 009
			Th[ 2 ]: 04 = 016
			Th[ 2 ]: 05 = 025
			Th[ 3 ]: 06 = 036
			Th[ 3 ]: 07 = 049
			Th[ 0 ]: 08 = 064
			Th[ 0 ]: 09 = 081
			Th[ 1 ]: 10 = 100
			Th[ 1 ]: 11 = 121
			Th[ 2 ]: 12 = 144
			Th[ 2 ]: 13 = 169
			Th[ 3 ]: 14 = 196
			Th[ 3 ]: 15 = 225
		\end{verbatim}
	\end{multicols}
 
 \subsection{Problemas encontrados} 
 	O erro no programa anterior está em usar a \textbf{diretiva  ordered} sem colocar um \textbf{bloco ordered} dentro de for (linhas 9-10) ~\cite{Lawrence14}. Por isso que ao executar o programa, o resultado será similar a:
	\begin{multicols}{3}
		\begin{verbatim} 
			Th[ 2 ]: 04 = 016
			Th[ 0 ]: 00 = 000
			Th[ 3 ]: 06 = 036
			Th[ 1 ]: 02 = 004
			Th[ 2 ]: 05 = 025
			Th[ 0 ]: 01 = 001
			Th[ 3 ]: 07 = 049
			Th[ 1 ]: 03 = 009
			Th[ 0 ]: 08 = 064
			Th[ 1 ]: 10 = 100
			Th[ 0 ]: 09 = 081
			Th[ 2 ]: 12 = 144
			Th[ 3 ]: 14 = 196
			Th[ 1 ]: 11 = 121
			Th[ 2 ]: 13 = 169
			Th[ 3 ]: 15 = 225
		\end{verbatim}
	\end{multicols}

\subsection{Correções} 
	Para que o resultado do programa seja correto deve ser adicionado o \textbf{bloco ordered} e neste caso só é necessário colocá-lo para a linha 10 porque quer-se imprimir em ordem, mas também poderia ser colocado para as linhas 9 e 10.
	\begin{lstlisting} 
		#include <stdio.h>
		#include <omp.h>
		#define SIZE 16

		int main(){
			int A[ SIZE ] , i ;
			#pragma omp parallel for schedule( static , 2 ) num_threads( 4 ) ordered
			for( i = 0 ; i < SIZE ; i++){
				A[ i ] = i * i ;
				#pragma omp ordered
				{
					printf( "Th[ %d ]: %02d = %03d\n" , omp_get_thread_num() , i , A[ i ] ) ;
				}
			}
			return 0 ;
		}
	\end{lstlisting}
 
\subsection{Conclusões} 
	A diretiva ordered deve ser sempre usada com o bloco ordered para que o compilador pode saber quais tarefas devem ser executadas em ordem, caso contrário, os resultados poderiam não ser corretos. Quando é bem utilizada, esta diretiva garante que só um thread por bloco ordered esteja em execução enquanto os outros aguardam ele ~\cite{Lawrence14}.

\clearpage 
 \section{Exercício 2} 
\label{sec:tab} 
\textbf{Modifique o programa mult.c, que realiza a multiplicação de 2 matrizes. Modifique este código para que ele realize a multiplicação utilizando as primitivas de paralelização de OpenMP. Compare o desempenho com 1, 2, 3 4, 8 e 16 threads. Tente realizar a paralelização no laço for das variáveis i, j e k, explicando no relatório se o comportamento obtido está correto ou não. Apresente e descreva no relatório grafos, tabelas e estatísticas dos tempos de execução.
\begin{itemize} 
\item Compare o desempenho obtido, explicando por que melhorou ou piorou e compare também a execução do programa em sua versão sequencial. 
\item Um dos objetivos e verificar se algum overhead e inserido pelo ambiente de execução (runtime OpenMP) quando a versão paralela do programa em OpenMP executa apenas com uma (1) thread, comparando-se com a versão sequencial (mult.c). 
\item Esse programa deve ser executado em pelo menos dois processadores diferentes, para efeitos de encontrar os intervalos de confiança cada execução deve ser repetida pelo menos 10 vezes. 
\item Outras informações que julgarem pertinentes ao contexto do trabalho podem ser adicionadas, e poderão ser somadas como pontos adicionais do EP.
\end{itemize}
}
\subsection{Experimentos} 
Os experimentos realizados neste exercício podem ser separados em duas categorias: a primeira consiste da adição de diretivas OpenMP para criar as zonas paralelas, a segunda consiste tanto da adição das  diretivas como de câmbios no mesmo código para otimizar o uso dos recursos. Cada uma de estas categorias conta com 4 programas diferentes: o programa sequencial, o programa com paralelização no laço da variável $i$, o programa com paralelização no laço da variável $j$, e finalmente, o programa com paralelização no laço da variável $k$. Cada programa, a excepção do programa sequencial, foi executado com diferentes números de threads.  
 
Algumas das características importantes dos experimentos são: 
\begin{itemize} 
    \item Otimização de código. Certas partes do programa original mult.c foram trocadas para melhorar o uso dos recursos. Estas otimizações foram dois: 
    \begin{itemize}
        \item Variável temporária. A variável temporária, $tmp$, foi criada para armazenar a suma das multiplicações no laço $k$.
        \item Método de transposição. Um método de transposição foi implementado para obter um acesso sequencial da matriz $b$ no laço $k$.
    \end{itemize}
    \item Zonas paralelas. Diferentes zonas foram criadas nos laços das variáveis $i$, $j$ e $k$.  
    \item Número de threads. O desempenho foi avaliado com 1, 2, 3, 4, 8 e 16 threads.   
    \item Tipo de programação do $for$. Foi empregado o tipo $static$ pois o conteúdo de cada um dos laços é semelhante um do outro e por tanto o overhead para determinar qual dos threads vai executar a próxima carga de trabalho não parece ser necessária.
    \item Processadores. Foram utilizados o: \textbf{Intel Core i5-3317U CPU @ 1.70GHz x 4} e o \textbf{Intel Xeon CPU E5-2630L v2 @ 2.40GHz}.
\end{itemize} 

O código de cada um dos experimentos pode ser visto nos Anexo 1 e 2. Sendo o Anexo 1 o conjunto de códigos com adições de diretivas de paralelização; e o Anexo 2 os códigos com diretivas de paralelização, além das otimizações mencionadas previamente.  

\subsection{Resultados} 

Os resultados nos tempos de execução dos experimentos da primeira categoria, i.e. os programas com diretivas OpenMP sem otimizações, podem ser vistos nas tabelas \ref{tabcorei5so} e \ref{tabxeonso}: 

\begin{table}[!htb]
    \begin{subtable}{.5\linewidth}
        \centering{
        \resizebox{.9\columnwidth}{!}{%
        \begin{tabular}{|c|r|r|r|r|} \hline
            \multicolumn{1}{|c|}{{\# Threads}} & \multicolumn{1}{|c|}       {{Sequencial}} & \multicolumn{1}{|c|}{{Paralela $i$}} & \multicolumn{1}{|c|}{{Paralela $j$}} & \multicolumn{1}{|c|}{{Paralela $k$}} \\ \cline{1-5}
            1    & 1.4042 & 0.7065  & 0.7559  & 0.7025  \\\hline
            2    & 1.4042 & 0.7169  & 0.7603  & 0.6920  \\\hline
            3    & 1.4042 & 0.7171  & 0.7662  & 0.7268  \\\hline
            4    & 1.4042 & 0.7137  & 0.7616  & 0.6976  \\\hline
            8    & 1.4042 & 0.7074  & 0.7573  & 0.7060  \\\hline
            16   & 1.4042 & 0.7129  & 0.7566  & 0.6960  \\\hline
        \end{tabular}%%
        }}
        \caption{Intel Core i5}
        \label{tabcorei5so}
    \end{subtable}%
    \begin{subtable}{.5\linewidth}
        \centering{
        \resizebox{.9\columnwidth}{!}{%
        \begin{tabular}{|c|r|r|r|r|} \hline
            \multicolumn{1}{|c|}{{\# Threads}} & \multicolumn{1}{|c|}{{Sequencial}} & \multicolumn{1}{|c|}{{Paralela $i$}} & \multicolumn{1}{|c|}{{Paralela $j$}} & \multicolumn{1}{|c|}{{Paralela $k$}} \\ \cline{1-5}
            1    & 2.3469 & 1.2561  & 1.2747  & 0.6990  \\\hline
            2    & 2.3469 & 1.1427  & 1.1572  & 0.6424  \\\hline
            3    & 2.3469 & 1.1389  & 1.1804  & 0.6402  \\\hline
            4    & 2.3469 & 1.2746  & 1.1563  & 0.8217  \\\hline
            8    & 2.3469 & 1.1286  & 1.1179  & 0.7177  \\\hline
            16   & 2.3469 & 1.1921  & 1.1979  & 0.6943  \\\hline
        \end{tabular}%
        }}
        \caption{Intel Xeon}
        \label{tabxeonso}
    \end{subtable}
    \caption{Diretivas OpenMP, sem Otimizações}
    \label{tabso}
\end{table}

Os gráficos dos tempos de execução previamente presentados podem ser vistos nas figuras ~\ref{fig:corei5so} e ~\ref{fig:xeonso}.
 
\begin{figure}[htb] 
    \centering
    \begin{subfigure}{.5\textwidth}
        \centering
        \includegraphics[height=6cm]{Images/corei5so} 
        \caption{Intel Core i5}
        \label{fig:corei5so} 
    \end{subfigure}%
    \begin{subfigure}{.5\textwidth}
        \centering
        \includegraphics[height=6cm]{Images/xeonso} 
        \caption{Intel Xeon}
        \label{fig:xeonso} 
    \end{subfigure}
    \caption{Diretivas OpenMP, sem otimizações}
    \label{fig:so}
\end{figure} 

Em relação à quantidade de threads, é visível que no caso da Core i5 os tempos se mantém semelhantes; porém, no caso da Xeon os melhores tempos são obtidos quando o número de threads é 2-3. Também pode-se perceber que o tempo aumentou de maneira significativa quando o número de threads foi 4. O que podemos concluir de estes resultados é que no caso da Core i5, que possui 4 cores, os tempos não mudaram de maneira significativa pois o hardware está preparado para a paralelização. No caso da Xeon, que também possui 4 cores, os tempos mudam mais significativamente, provavelmente devido ao overhead pela mudança de contexto. O overhead pelo uso das diretivas não foi significativo neste caso pois ainda com um thread o tempo é menor em ambos processadores na versão com diretivas OpenMP do que na versão sequencial.

Em relação aos laços onde a paralelização foi efetuada, pode ser enxergado que a paralelização no laço da variável $k$ apresenta um tempo menor de execução. Isto pode ser devido ao uso de uma variável temporária $temp$ necessária para poder realizar a paralelização. Ao adicionar esta variável se libera a restrição de ter que acessar à matriz $c$ cada vez que se quer adicionar uma multiplicação. Além disso, em ambos processadores, os tempos de execução são menores quando a paralelização encontra-se no laço da variável $i$, uma única excepção é no caso da Xeon com 4 threads. Isto é porque quando o laço $i$ é paralelizado a carga de trabalho é significativa em comparação aos outros laços.

Os resultados nos tempos de execução dos experimentos da segunda categoria, i.e. os programas com diretivas OpenMP e com otimizações, podem ser vistos nas tabelas \ref{tabcorei5co} e \ref{tabxeonco}: 
\begin{table}[!htb]
    \centering
    \begin{subtable}{.5\linewidth}
        \centering{
        \resizebox{.9\columnwidth}{!}{%
        \begin{tabular}{|c|r|r|r|r|} \hline
            \multicolumn{1}{|c|}{{\# Threads}} & \multicolumn{1}{|c|}{{Sequencial}} & \multicolumn{1}{|c|}{{Paralela $i$}} & \multicolumn{1}{|c|}{{Paralela $j$}} & \multicolumn{1}{|c|}{{Paralela $k$}} \\ \cline{1-5}
            1    & 0.5229 & 0.3506  & 0.4999  & 0.6410  \\\hline
            2    & 0.5229 & 0.3479  & 0.4848  & 0.6375  \\\hline
            3    & 0.5229 & 0.3519  & 0.4873  & 0.6349  \\\hline
            4    & 0.5229 & 0.3500  & 0.3703  & 0.6383  \\\hline
            8    & 0.5229 & 0.3486  & 0.3468  & 0.6361  \\\hline
            16   & 0.5229 & 0.3957  & 0.3477  & 0.6318  \\\hline
        \end{tabular}%
        }}
        \caption{Intel Core i5}
        \label{tabcorei5co}
    \end{subtable}%
    \begin{subtable}{.5\linewidth}
        \centering{
        \resizebox{.9\columnwidth}{!}{%
        \begin{tabular}{|c|r|r|r|r|} \hline
            \multicolumn{1}{|c|}{{\# Threads}} & \multicolumn{1}{|c|}{{Sequencial}} & \multicolumn{1}{|c|}{{Paralela $i$}} & \multicolumn{1}{|c|}{{Paralela $j$}} & \multicolumn{1}{|c|}{{Paralela $k$}} \\ \cline{1-5}
            1    & 0.5411 & 0.3448  & 0.3306  & 0.5571  \\\hline
            2    & 0.5411 & 0.3225  & 0.3341  & 0.5535  \\\hline
            3    & 0.5411 & 0.3294  & 0.3528  & 0.5808  \\\hline
            4    & 0.5411 & 0.3431  & 0.3650  & 0.5684  \\\hline
            8    & 0.5411 & 0.3351  & 0.3316  & 0.5383  \\\hline
            16   & 0.5411 & 0.3546  & 0.3274  & 0.5904  \\\hline
        \end{tabular}%
        }}
        \caption{Intel Xeon}
        \label{tabxeonco}
    \end{subtable}
    \caption{Diretivas OpenMP, com otimizações}
    \label{tabco}
\end{table}

Os gráficos dos tempos de execução podem ser vistos nas figuras ~\ref{fig:corei5co} e ~\ref{fig:xeonco}. 

\begin{figure} 
    \centering
    \begin{subfigure}{.5\textwidth}
        \centering
        \includegraphics[height=6cm]{Images/corei5co} 
        \caption{Intel Core i5}
        \label{fig:corei5co} 
    \end{subfigure}%
    \begin{subfigure}{.5\textwidth}
        \centering
        \includegraphics[height=6cm]{Images/xeonco} 
        \caption{Intel Xeon}
        \label{fig:xeonco} 
    \end{subfigure}    
    \caption{Diretivas OpenMP, com otimizações}
    \label{fig:co}
\end{figure} 

Uma das primeiras conclusões que pode-se enxergar é: Ao gerar código tendo em consideração como a memória é acessada, o tempo de execução dele pode diminuir de maneira considerável. 

Em ambos processadores é percebível que o tempo de execução é menor em comparação a suas contra partes sem otimizações de código. Para avaliar melhor os gráficos dos tempos obtidos neste caso se geraram as figuras ~\ref{fig:corei5cozoom} e ~\ref{fig:xeoncozoom}, onde o rango do eixo $y$ é menor. 

\begin{figure} 
    \centering
    \begin{subfigure}{.5\textwidth}
        \centering
        \includegraphics[height=6cm]{Images/corei5cozoom} 
        \caption{Intel Core i5}
        \label{fig:corei5cozoom} 
    \end{subfigure}%
    \begin{subfigure}{.5\textwidth}
        \centering
        \includegraphics[height=6cm]{Images/xeoncozoom} 
        \caption{Intel Xeon}
        \label{fig:xeoncozoom} 
    \end{subfigure}    
    \caption{Diretivas OpenMP, com otimizações (Zoom)}
    \label{fig:cozoom}
\end{figure} 

Em relação à quantidade de threads, é visível que no caso da Core i5 os tempos se mantém semelhantes no caso da paralelização do laço da varável $k$; porém, no laço $i$ pode-se enxergar uma grande diminuição no tempo quando 4 threads são usados, com 8 e 16 o tempo diminui um pouco mas. No caso da Xeon os melhores tempos são obtidos quando o número de threads é 2 ou 8. O que podemos concluir de estes resultados é que a quantidade de threads vira a ser um aspecto importante e que o melhor valor é 8, isto e possivelmente devido a que os processadores contam com 4 cores, mas no caso da Core i5  também conta com tecnologia hyper-threading. O overhead pelo uso das diretivas só é significativo em comparação à paralelização no laço $k$.

Em relação aos laços onde a paralelização foi efetuada, pode ser enxergado que a paralelização no laço da variável $k$ apresenta um tempo maior de execução. Isto é devido ao uso de uma variável temporária $tmp$ em todas as paralelizações. Agora que todas as paralelizações contam com esta mesma otimização pode se observar que o overhead gerado pela mudança de contexto é maior quando a paralelização é gerado no laço mais interno. Além disso, em ambos processadores, os tempos de execução são menores quando a paralelização encontra-se no laço da variável $i$ ou $j$. Sendo os menores tempos obtidos quando a quantidade de threads é 8 e os tempos não apresentam variância significativa entre a paralelização no laço $i$ e o laço $j$. 

\subsection{Conclusões} 
O que pode-se concluir dos experimentos é que a paralelização ajuda na melhoria dos tempos de execução. Porém, as otimizações de código para facilitar a paralelização e o acesso na memoria podem gerar melhores resultados. O tempo de execução da versão sequencial (sem diretivas OpenMP) foi menor do que os tempos das versões paralelizadas sem otimizações. 

Além disso, o overhead pela mudança de contexto é maior quando a paralelização é criada nos laços internos, onde a carga de trabalho é menor. Então, os threads executam o trabalho de maneira rápida, mas devem liberar os recursos para que outro thread possa trabalhar. Isto se repete constantemente, pois cada thread tem uma carga de trabalaho pequena, e isto gera o overhead. 

No aso do número de threads, ao empregar 8, os tempos foram melhores no Core i5. Isto pode ser explicado pelo fato de que o Core i5 consiste de 4 cores mas também usa a tecnologia hyper-threading, o que gera que o sistema operativo assuma que conta com 8 cores e assim pode distribuir melhor os threads requeridos pelo programa. 

\clearpage 

\section{Conclusões} 
Nossas conclusões finais são:
\begin{itemize}
    \item É muito importante que se tenha em consideração o modo em que a memoria é acessada para garantir o melhor tempo de execução.
    \item Para evitar o overhead, a carga de cada um dos threads debe ser considerável. Sempre que ficar na dúvida, o laço exterior é a melhor opção. 
    \item Para a escolha do número de threads é importante conhecer a arquitetura no qual será executado o programa. Quando esta informação não é conhecida, a melhor opção é assumir que o tempo não será o ótimo.
    \item Na uso das diretivas OpenMP é importante conhecer tudo o que deve ser colocado no código (como o caso da diretiva ordered) para obter os resultados corretos.
\end{itemize}
 
\clearpage 
 
\bibliographystyle{plain}   
\bibliography{bibliografia.bib} 
\appendix 
 
\section{Anexo I} 
\label{anex1} 
 
Código do programa mult.c sequencial com trocas para a medição do tempo. 
\begin{verbatim} 
/*
ID: diana.n1
PROG: MultSeq
LANG: C++
*/

#include <stdio.h>
#include <assert.h>
#include <stdlib.h>
#include <assert.h>
#include <time.h>
#include "omp.h"


int main(int argc, char **argv) {
    double start, end; 
    start = omp_get_wtime();
    long **a, **b, **c;
    int N = 500;
	
    if (argc == 2) {
        N = atoi (argv[1]);
        assert (N > 0);
    }

    int i,j,k,mul=5;
    long col_sum = N * (N-1) / 2;
	
			
	    a = (long **)malloc (N * sizeof(long *));
	    b = (long **)malloc (N * sizeof(long *));
	    c = (long **)malloc (N * sizeof(long *));
	
	    for (i=0; i<N; i++) {
	        a[i] = (long *)malloc (N * sizeof(long));
	        b[i] = (long *)malloc (N * sizeof(long));
	        c[i] = (long *)malloc (N * sizeof(long));
	    }


	    for (i=0; i<N; i++)
	        for (j=0; j<N; j++) {
		        a[i][j] = i*mul;
	        	b[i][j] = i;
		        c[i][j] = 0;
	        }

	    printf ("Matrix generation finished.\n");         
	
	    for (i=0; i<N; i++)
	        for (j=0; j<N; j++)
		        for (k=0; k<N; k++)
		        	c[i][j] += a[i][k] * b[k][j];

	    printf ("Multiplication finished.\n");         
	
	    for (i=0; i<N; i++)
	        for (j=0; j<N; j++)
		        assert ( c[i][j] == i*mul * col_sum);  
        printf ("Test finished.\n");         

	    end = omp_get_wtime();
	    printf("Time: %lf.\n", end-start);
}
\end{verbatim}
 
Código do programa mult.c com paralelização no laço da variável i. 
\begin{verbatim} 
/*
ID: diana.n1
PROG: MultParalleli
LANG: C++
*/

#include <stdio.h>
#include <assert.h>
#include <stdlib.h>
#include <assert.h>
#include <time.h>
#include "omp.h"

int main(int argc, char **argv) {
	    double start, end; 
	    start = omp_get_wtime();
	    int nthreads, tid, chunk = 10;
	
	    long **a, **b, **c;
	    int N = 500;
	
    if (argc == 2) {
        N = atoi (argv[1]);
        assert (N > 0);
    }

    int i,j,k,mul=5;
    long col_sum = N * (N-1) / 2;
	
	    a = (long **)malloc (N * sizeof(long *));
	    b = (long **)malloc (N * sizeof(long *));
	    c = (long **)malloc (N * sizeof(long *));
	
	    for (i=0; i<N; i++) {
	        a[i] = (long *)malloc (N * sizeof(long));
	        b[i] = (long *)malloc (N * sizeof(long));
	        c[i] = (long *)malloc (N * sizeof(long));
	    }

	    #pragma omp parallel shared (a,b,c,nthreads, chunk) private (i, j,k,tid) 
	    {
        tid = omp_get_thread_num();
        #pragma omp single
            printf("Number threads = %d\n", omp_get_num_threads());

        #pragma omp for schedule(static, chunk)
            for (i=0; i<N; i++){
                for (j=0; j<N; j++) {
                    a[i][j] = i*mul;
                    b[i][j] = i;
                    c[i][j] = 0;
                }
            }
		
        #pragma omp single
            printf ("Matrix generation finished.\n");         
		
        #pragma omp for schedule(static, chunk)
            for (i=0; i<N; i++){
                for (j=0; j<N; j++){
                    for (k=0; k<N; k++){
                        c[i][j] += a[i][k] * b[k][j];
                    }
                }
            }
		
        #pragma omp single
            printf ("Multiplication finished.\n");         
	
        #pragma omp for schedule(static, chunk)
            for (i=0; i<N; i++){
                for (j=0; j<N; j++){
                    assert ( c[i][j] == i*mul * col_sum);  
                }
            }
        #pragma omp single
            printf ("Test finished.\n");         
    }
    end = omp_get_wtime();
    printf("Time: %lf.\n", end-start);
}

\end{verbatim} 
 
Código do programa mult.c com paralelização no laço da variável j. 

\begin{verbatim} 
/*
ID: diana.n1
PROG: MultParallelj
LANG: C++
*/

#include <stdio.h>
#include <assert.h>
#include <stdlib.h>
#include <assert.h>
#include <time.h>
#include "omp.h"

int main(int argc, char **argv) {
    double start, end; 
    start = omp_get_wtime();
    int nthreads, tid, chunk = 10;
	
    long **a, **b, **c;
    int N = 500;
	
    if (argc == 2) {
        N = atoi (argv[1]);
        assert (N > 0);
    }

    int i,j,k,mul=5;
    long col_sum = N * (N-1) / 2;
	
    a = (long **)malloc (N * sizeof(long *));
    b = (long **)malloc (N * sizeof(long *));
    c = (long **)malloc (N * sizeof(long *));
	
    for (i=0; i<N; i++) {
        a[i] = (long *)malloc (N * sizeof(long));
        b[i] = (long *)malloc (N * sizeof(long));
        c[i] = (long *)malloc (N * sizeof(long));
    }

	    #pragma omp parallel shared (a,b,c,nthreads, chunk) private (i, j,k,tid) 
	    {
		        tid = omp_get_thread_num();
		        #pragma omp single
			            printf("Number threads = %d\n", omp_get_num_threads());

		        #pragma omp for schedule(static, chunk)
		            for (i=0; i<N; i++){
			                for (j=0; j<N; j++) {
				                    a[i][j] = i*mul;
				                    b[i][j] = i;
				                    c[i][j] = 0;
	                  	}
		            }
		        #pragma omp single
    			        printf ("Matrix generation finished.\n");         
		
		        for (i=0; i<N; i++){
	    		        #pragma omp for schedule(static, chunk)
	    			            for (j=0; j<N; j++){
	    				                for (k=0; k<N; k++){
	    					                    c[i][j] += a[i][k] * b[k][j];
	    				                }
	    	          		 }
	    	    }
		
		        #pragma omp single
			            printf ("Multiplication finished.\n");         
	
		        #pragma omp for schedule(static, chunk)
			            for (i=0; i<N; i++){
			          	    for (j=0; j<N; j++){
					                    assert ( c[i][j] == i*mul * col_sum);  
				                }
			        }
		        #pragma omp single
			            printf ("Test finished.\n");         
	    }

	    end = omp_get_wtime();
	    printf("Time: %lf.\n", end-start);
}

\end{verbatim} 
 
Código do programa mult.c com paralelização no laço da variável k. 
\begin{verbatim} 
/*
ID: diana.n1
PROG: MultParallelk
LANG: C++
*/

#include <stdio.h>
#include <assert.h>
#include <stdlib.h>
#include <assert.h>
#include <time.h>
#include "omp.h"

int main(int argc, char **argv) {
    double start, end; 
    start = omp_get_wtime();
    int nthreads, tid, chunk = 10;
	
    long **a, **b, **c;
    int N = 500;
	
    if (argc == 2) {
        N = atoi (argv[1]);
        assert (N > 0);
    }

    int i,j,k,mul=5;
    long col_sum = N * (N-1) / 2;
    long temp;
	
    a = (long **)malloc (N * sizeof(long *));
    b = (long **)malloc (N * sizeof(long *));
    c = (long **)malloc (N * sizeof(long *));
	
    for (i=0; i<N; i++) {
        a[i] = (long *)malloc (N * sizeof(long));
        b[i] = (long *)malloc (N * sizeof(long));
        c[i] = (long *)malloc (N * sizeof(long));
    }

    #pragma omp parallel shared (a,b,c,nthreads, chunk, temp) private (i, j,k,tid)
    {
        tid = omp_get_thread_num();
        #pragma omp single
            printf("Number threads = %d\n", omp_get_num_threads());

        #pragma omp for schedule(static, chunk)
        for (i=0; i<N; i++){
            for (j=0; j<N; j++) {
                a[i][j] = i*mul;
                b[i][j] = i;
                c[i][j] = 0;
            }
        }
		
        #pragma omp single
            printf ("Matrix generation finished.\n");         
		
        for (i=0; i<N; i++){
            for (j=0; j<N; j++){
                #pragma omp single
                    temp = 0;
                #pragma omp for schedule(static, chunk) reduction (+:temp) 
                    for (k=0; k<N; k++){
                        temp += a[i][k] * b[k][j];
                    }
                #pragma omp single
                    c[i][j] = temp;
            }
        }
		
        #pragma omp single
            printf ("Multiplication finished.\n");         
	
        #pragma omp for schedule(static, chunk)
            for (i=0; i<N; i++){
                for (j=0; j<N; j++){
                    assert ( c[i][j] == i*mul * col_sum);  
                }
            }
        #pragma omp single
            printf ("Test finished.\n");         
    }
    end = omp_get_wtime();
    printf("Time: %lf.\n", end-start);
}

\end{verbatim} 
 
\section{Anexo II} 
\label{anex2} 
 
Código do programa mult.c sequencial com optimizações de código. 
\begin{verbatim} 
/*
ID: diana.n1
PROG: MultSeqCodeChange
LANG: C++
*/

#include <stdio.h>
#include <assert.h>
#include <stdlib.h>
#include <assert.h>
#include <time.h>
#include "omp.h"

void transpose(int n, long ** m){
    long tmp;
    for (int i = 0; i < n; i++){
        for (int j = i+1; j < n; j++){
            tmp = m[i][j];
            m[i][j] = m[j][i];
            m[j][i] = tmp;
        }
    }	
}

int main(int argc, char **argv) {
    double start, end; 
    start = omp_get_wtime();
    long **a, **b, **c;
    int N = 500;
	
    if (argc == 2) {
        N = atoi (argv[1]);
        assert (N > 0);
    }

    int i,j,k,mul=5;
    long col_sum = N * (N-1) / 2, tmp;
	
		
    a = (long **)malloc (N * sizeof(long *));
    b = (long **)malloc (N * sizeof(long *));
    c = (long **)malloc (N * sizeof(long *));
    for (i=0; i<N; i++) {
        a[i] = (long *)malloc (N * sizeof(long));
        b[i] = (long *)malloc (N * sizeof(long));
        c[i] = (long *)malloc (N * sizeof(long));
    }


    for (i=0; i<N; i++){
        for (j=0; j<N; j++) {
            a[i][j] = i*mul;
            b[i][j] = i;
            c[i][j] = 0;
        }
    }

    printf ("Matrix generation finished.\n");         

    transpose(N, b);
    for (i=0; i<N; i++){
      	for (j=0; j<N; j++){
            tmp = 0;
            for (k=0; k<N; k++){
                tmp += a[i][k] * b[j][k];
            }
            c[i][j] = tmp;
      	}
    }

    printf ("Multiplication finished.\n");         

    for (i=0; i<N; i++)
        for (j=0; j<N; j++)
            assert ( c[i][j] == i*mul * col_sum);  
    
    printf ("Test finished.\n");         

    end = omp_get_wtime();
    printf("Time: %lf.\n", end-start);
}

\end{verbatim} 
 
Código do programa mult.c com paralelização no laço da variável i com optimizações de código. 
\begin{verbatim} 

/*
ID: diana.n1
PROG: MultParalleliCodeChange
LANG: C++
*/

#include <stdio.h>
#include <assert.h>
#include <stdlib.h>
#include <assert.h>
#include <time.h>
#include "omp.h"

void transpose(int n, long ** m){
    long tmp;
    for (int i = 0; i < n; i++){
        for (int j = i+1; j < n; j++){
            tmp = m[i][j];
            m[i][j] = m[j][i];
            m[j][i] = tmp;
        }
    }	
}

int main(int argc, char **argv) {
    double start, end; 
    start = omp_get_wtime();
    int nthreads, tid, chunk = 10;
	
    long **a, **b, **c;
    int N = 500;
	
    if (argc == 2) {
        N = atoi (argv[1]);
        assert (N > 0);
    }

    int i,j,k,mul=5;
    long col_sum = N * (N-1) / 2, tmp;
	
    a = (long **)malloc (N * sizeof(long *));
    b = (long **)malloc (N * sizeof(long *));
    c = (long **)malloc (N * sizeof(long *));

    for (i=0; i<N; i++) {
        a[i] = (long *)malloc (N * sizeof(long));
        b[i] = (long *)malloc (N * sizeof(long));
        c[i] = (long *)malloc (N * sizeof(long));
    }

    #pragma omp parallel shared (a,b,c,nthreads, chunk) private (i, j,k,tid, tmp) 
    {
        tid = omp_get_thread_num();
	
        #pragma omp single
            printf("Number threads = %d\n", omp_get_num_threads());

        #pragma omp for schedule(static, chunk)
            for (i=0; i<N; i++){
                for (j=0; j<N; j++) {
                    a[i][j] = i*mul;
                    b[i][j] = i;
                    c[i][j] = 0;
                }
            }

        #pragma omp single
            printf ("Matrix generation finished.\n");         

        #pragma omp single
            transpose(N, b);

        #pragma omp for schedule(static, chunk)
            for (i=0; i<N; i++){
                for (j=0; j<N; j++){	
                    tmp = 0;
                    for (k=0; k<N; k++){
                        tmp += a[i][k] * b[j][k];
                    }
                    c[i][j] = tmp;
                }
            }

        #pragma omp single
            printf ("Multiplication finished.\n");         

        #pragma omp for schedule(static, chunk)
            for (i=0; i<N; i++){
              	for (j=0; j<N; j++){
                    assert ( c[i][j] == i*mul * col_sum);  
                }
            }
        #pragma omp single
            printf ("Test finished.\n");         
    }
    end = omp_get_wtime();
    printf("Time: %lf.\n", end-start);
}

\end{verbatim} 
 
Código do programa mult.c com paralelização no laço da variável j com optimizações de código. 
\begin{verbatim} 
/*
ID: diana.n1
PROG: MultParalleljCodeChange
LANG: C++
*/

#include <stdio.h>
#include <assert.h>
#include <stdlib.h>
#include <assert.h>
#include <time.h>
#include "omp.h"

void transpose(int n, long ** m){
    long tmp;
    for (int i = 0; i < n; i++){
        for (int j = i+1; j < n; j++){
            tmp = m[i][j];
            m[i][j] = m[j][i];
            m[j][i] = tmp;
        }
    }	
}

int main(int argc, char **argv) {
    double start, end; 
    start = omp_get_wtime();
    int nthreads, tid, chunk = 10;
	
    long **a, **b, **c;
    int N = 500;
	
    if (argc == 2) {
        N = atoi (argv[1]);
        assert (N > 0);
    }

    int i,j,k,mul=5;
    long col_sum = N * (N-1) / 2, tmp;
	
    a = (long **)malloc (N * sizeof(long *));
    b = (long **)malloc (N * sizeof(long *));
    c = (long **)malloc (N * sizeof(long *));

    for (i=0; i<N; i++) {
        a[i] = (long *)malloc (N * sizeof(long));
        b[i] = (long *)malloc (N * sizeof(long));
        c[i] = (long *)malloc (N * sizeof(long));
    }

    #pragma omp parallel shared (a,b,c,nthreads, chunk) private (i,  j, k, tid, tmp) 
    {
        tid = omp_get_thread_num();
        #pragma omp single
            printf("Number threads = %d\n", omp_get_num_threads());

        #pragma omp for schedule(static, chunk)
            for (i=0; i<N; i++){
                for (j=0; j<N; j++) {
                    a[i][j] = i*mul;
                    b[i][j] = i;
                    c[i][j] = 0;
                }
            }
        #pragma omp single
            printf ("Matrix generation finished.\n");      

        #pragma omp single
            transpose(N, b);

        for (i=0; i<N; i++){
            #pragma omp for schedule(static, chunk)
                for (j=0; j<N; j++){
                    tmp = 0;
                    for (k=0; k<N; k++){
                        tmp += a[i][k] * b[j][k];
                    }
                    c[i][j] = tmp;
                }
        }

        #pragma omp single
            printf ("Multiplication finished.\n");         

        #pragma omp for schedule(static, chunk)
            for (i=0; i<N; i++){
              	for (j=0; j<N; j++){
                    assert ( c[i][j] == i*mul * col_sum);  
                }
            }
        #pragma omp single
            printf ("Test finished.\n");         
    }
    end = omp_get_wtime();
    printf("Time: %lf.\n", end-start);
}

\end{verbatim} 
 
Código do programa mult.c com paralelização no laço da variável k com optimizações de código. 
\begin{verbatim} 
/*
ID: diana.n1
PROG: MultParallelkCodeChange
LANG: C++
*/

#include <stdio.h>
#include <assert.h>
#include <stdlib.h>
#include <assert.h>
#include <time.h>
#include "omp.h"

void transpose(int n, long ** m){
    long tmp;
    for (int i = 0; i < n; i++){
        for (int j = i+1; j < n; j++){
            tmp = m[i][j];
            m[i][j] = m[j][i];
            m[j][i] = tmp;
        }
    }	
}

int main(int argc, char **argv) {
    double start, end; 
    start = omp_get_wtime();
    int nthreads, tid, chunk = 10;

    long **a, **b, **c;
    int N = 500;

    if (argc == 2) {
        N = atoi (argv[1]);
        assert (N > 0);
    }

    int i,j,k,mul=5;
    long col_sum = N * (N-1) / 2, tmp;

    a = (long **)malloc (N * sizeof(long *));
    b = (long **)malloc (N * sizeof(long *));
    c = (long **)malloc (N * sizeof(long *));

    for (i=0; i<N; i++) {
        a[i] = (long *)malloc (N * sizeof(long));
        b[i] = (long *)malloc (N * sizeof(long));
        c[i] = (long *)malloc (N * sizeof(long));
    }

    #pragma omp parallel shared (a,b,c,nthreads, chunk, tmp) private (i, j,k,tid)
    {
        tid = omp_get_thread_num();
        #pragma omp single
            printf("Number threads = %d\n", omp_get_num_threads());

        #pragma omp for schedule(static, chunk)
            for (i=0; i<N; i++){
                for (j=0; j<N; j++) {
                    a[i][j] = i*mul;
                    b[i][j] = i;
                    c[i][j] = 0;
          		}
            }
        #pragma omp single
            printf ("Matrix generation finished.\n");         

        #pragma omp single
            transpose(N, b);

        for (i=0; i<N; i++){
            for (j=0; j<N; j++){
                #pragma omp single
                    tmp = 0;
                #pragma omp for schedule(static, chunk) reduction (+:tmp) 
                    for (k=0; k<N; k++){
                        tmp += a[i][k] * b[j][k];
                    }
                #pragma omp single
                    c[i][j] = tmp;
          	}
        }

        #pragma omp single
            printf ("Multiplication finished.\n");         

        #pragma omp for schedule(static, chunk)
            for (i=0; i<N; i++){
              	for (j=0; j<N; j++){
                    assert ( c[i][j] == i*mul * col_sum);  
                }
            }
        #pragma omp single
            printf ("Test finished.\n");         
    }
    end = omp_get_wtime();
    printf("Time: %lf.\n", end-start);
}
\end{verbatim} 
 
\end{document} 
 
 
 
 
 
